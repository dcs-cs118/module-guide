\section{Introduction}

\subsection{What is this course all about?}

The Programming for Computer Scientists module is designed to give you knowledge and confidence in using a computer as a scientific tool. Whether you are already an experienced programmer or have never programmed before, this module will challenge everybody in order to bring everyone on the same level. (\textbf{TODO}: fix this) You can find a wealth of information on the module website:
\begin{center}
	\url{http://go.warwick.ac.uk/cs118/}
\end{center}

During the course you will have a chance to work on control software. 
You will get to spend a good deal of time solving software problems; your
solutions to these problems will be implemented on a computer and run in
order for you to check the resulting behaviour. 

Programming does not just involve writing code in a programming language, but it consists of three important steps -- specification, implementation, and evaluation. The exercises in 
this course encourage you to look at each of these steps in turn. Each step 
must be carried out successfully if the software which you are going to 
create is to be correct. 
 
The coursework for this course are set in the context of simulating a robot travelling through a virtual maze. This is what you will program; you will be able to see your progress as the robot will either reach the target of the maze or not. 
 
By the end of this module you will have learnt many of the features and techniques needed for computer programming. The first half of the module is structured so that the necessary components of procedural programming are introduced. The concepts covered are applicable to a whole host of different programming languages. Emphasis is placed on writing correct, efficient and maintainable programs. The second half of the module develops on the earlier techniques but with particular application to object oriented programming. The construction of well-designed interfaces, and program encapsulation and abstraction are discussed. The module is based on a number of example programs and emphasis is placed on coursework with the aim of ensuring that the theory covered in lectures is reinforced by practical programming exercises. 

\pagebreak

\subsection{Activities}

As part of this module, you will be attending lectures in which you are introduced to programming. This will be done at a relatively quick pace and you will be expected to deepen your understanding through independent study. Meanwhile, weekly lab sessions will allow you to practice these skills and solidify your understanding with the help of the module organisers and teaching assistants.

\subsubsection{Lectures}

There are two lectures per week: Mondays at 11am-12pm in L3 and Fridays at 9-10am in R0.21. Lecture slides will be posted to the module website for you to review at any point. You may find that these do not make complete sense outside the lectures since they are only illustrating what we say in the lectures---so don't miss the lectures!
 
{\bf \lectureOne }

Note that there will be no CS118 lecture on the Friday in Week 1 as you will have an introductory session with our Director of Undergraduate Studies instead at the same time.

\subsubsection{Labs}
 
You are also required to attend one lab session per week\footnote{Don't be confused by the fact that your timetable might have more than one CS118 lab slot marked on it. All this means is that they have printed all the lab times. Which session you will attend will be announced in due course.}. 
The lab groups are organised through Tabula and you 
should check your allocation on there or on My Warwick to find out when your first lab is. 
 
The labs serve two purposes. Firstly they allow you to interact with the module organisers in a more direct way as well as teaching assistants who will also be experts in the subject area and who will be able to guide you through any difficulties. Make good use of available staff as they are there to assist you and answer any questions which you may have concerning this module. Secondly the labs are designed to allow you to test yourself, to make sure that you are keeping up with the module material. The labs are also good opportunities to get assistance on any aspect of the coursework.
 
Lab attendance is \emph{not} optional, but the problem sheets for each lab are. Your overall mark for this module will be based on the coursework and the exam.

\subsection{Assessment}

Assessment for this module is broken down into two parts. The coursework will account for 40\% of you final 
mark and the remaining 60\% of the marks are taken from your 
two-hour which usually takes place in Week 1 of Term 3.

\subsubsection{Coursework}

There are two pieces of coursework. The first coursework must be submitted via Tabula by \deadlineone and the second must be submitted via Tabula by \deadlinetwo. Your work will be assessed as follows:

Firstly, your programs will be tested automatically to ensure that they work as intended and checked for plagiarism after you have submitted them. You will then be required to come in to the labs where you will have to explain your code to one of the teaching assistants who will also question you about the code.

This may sound a little heavy, but I can assure you that if you have done the necessary study and produced your own independent solutions then you will not have any problems.

A record of your work will be stored on Tabula, which allows us to record your progress and give you feedback.

\subsection{Books}

Programming can be complicated business, but the internet
should have plenty of information to ease the learning process.
However, some of you may prefer to learn from a introductory text book.
Each year I try to select the most appropriate Java books for this course. 
In making the selection I try to ensure that the books will be understandable,
will reflect the range of talents and abilities of the people on the course,
and will be useful later in your degree. 

If I were going to buy a good book from new then I would pick up one
of:
\begin{itemize}

\item
{\em Introduction to Java Programming Comprehensive} 
by Y Daniel Liang, published by Prentice Hall, ISBN 0-13-222158-6 

\item 
{\em Beginning Java Programming -- The Object-Oriented Approach} by 
Bart Baesens, Aim\'{e}e Backiel and Seppe vanden Broucke, published by 
Wrox, ISBN 978-1-118-73949-5 

\end{itemize}

The Liang book follows the lectures very closely so is your best bet if you are feeling a bit unsure about this programming lark. 
 
You may already have a Java book, but you may wish to check that it is up-to-date since programming languages evolve over time so a Java book from ten years ago may not be representative of modern Java. There are enough differences to warrant splashing out and buying a new book if you like having a book. Otherwise, a quick search online can answer most questions.

\subsection{Web-based course material}

All the module material is available on the web at:

\begin{center}
	\url{http://go.warwick.ac.uk/cs118/}
\end{center}

We suggest you add this address to your bookmarks. Note that you can also replace \texttt{cs118} with the code of any other module to get to the respective module website.

It is strongly recommended that you spend several hours at a terminal during the first few weeks of term getting familiar with
Linux, the tools, etc. In any event, you should also be spending
several hours per week writing Java programs. Copy examples out
of your textbooks, customise them for your own use and experiment
by trying to write programs of your own. Learning to program is impossible without the practical experience that is gained only by  solving problems, implementing solutions, and verifying their correctness.

\subsection{How do I get help?}
\label{sec:how-to-get-help}

You will probably find that many of the issues which concern you are also experienced by others on the course. Try talking to your fellow students, teaching assistants, or the module organisers either in person, through Slack, or via email. It is likely that your peers will either have similar problems or will have solutions to these issues. The teaching assistants are usually pretty good at answering questions. 

When getting help for the coursework, be mindful not to ask for someone else to do your work for you as this could constitute plagiarism. The teaching assistants and module organisers will be able to guide you in the right direction without directly giving you the answers.

You are always welcome to come and find the module organisers in their offices or get in touch via email:
\begin{center}
	\begin{tabular}{|c|c|c|}
		\hline 
		\textbf{Name} & \textbf{Office} & \textbf{Email} \\ 
		\hline 
		Sara Kalvala & MSB5.29 & \href{mailto:sara.kalvala@warwick.ac.uk}{sara.kalvala@warwick.ac.uk} \\ 
		\hline 
		Michael B. Gale & CS3.25 & \href{mailto:m.gale@warwick.ac.uk}{m.gale@warwick.ac.uk} \\ 
		\hline 
	\end{tabular} 
\end{center}
Although we are happy for you to just pop-in whenever we have time, you can find our office hours on the department's website at 
\begin{center}
	\url{https://warwick.ac.uk/fac/sci/dcs/teaching/officehours/}
\end{center}
These office hours are times that we have specifically set aside in order to meet our students without the need for an appointment.

\subsection{And if it all starts to go wrong?}

You may get to a point in this course where you just don't know what is going on any more. The golden rule is not to panic! There are plenty of people around who are here to help you. Your personal tutor is usually a great point of contact if you feel that things are getting out of hand.   

\subsection{Acknowledgements}

The robot-maze software which you will use in your coursework has an interesting history. It started life at Queen Mary College, University of London as the brainchild of Ian Page. When Ian moved to Oxford he and Colin Turnbull made extensive rewrites and the robot-maze to this day exists as a means by which engineering students learn the C programming language. Kevin Parrott, Alison Noble, Andrew Zisserman and Prof. Stephen Jarvis ran this software largely untouched at Oxford for a number of years. The new Java version of the software is thanks to Phil Mueller from the University of Warwick (now himself at Oxford), and the new exercises are courtesy of Ioannis Verdelis (formerly of Warwick and now at Manchester University). I think you will agree that the result is a wonderfully interactive way in which to learn the art of programming.

Similarly, this module guide has evolved over the years from Professor Michael Luck's version using Pascal and has benefited from the contributions of many others since then, including Professor Stephen Jarvis, Dr Abhir Bhalerao, and Dr Steven Wright.

\vfill
