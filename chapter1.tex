% !TEX root = doc.tex

\section{Preface}

\subsection{What is this course all about?}

The Programming for Computer Scientists course is designed to give you 
knowledge and confidence in using a computer as a scientific tool. 
During the course you will have a chance to work on control software. 
You will get to spend a good deal of time solving software problems; your
solutions to these problems will be implemented on a computer and run in
order for you to check the resulting behaviour. 

Despite the slightly misleading title, Programming for Computer Scientists
is not simply about `programming'. Problem solving using computer software 
involves three important steps -- design, build and test. The exercises in 
this course encourage you to look at each of these steps in turn. Each step 
must be carried out successfully if the software which you are going to 
create is to be correct. 
 
The laboratory exercises for this course are set in the context of getting a 
robot to travel through a maze. This is what you will program; you will 
be able to see your progress as the robot will either reach the end of the 
maze or not. 
 
By the end of the course you will have learnt a number of skills: You will 
have gained a great deal of experience in the art of software development,
you will also know what it means to plan and develop sophisticated Java 
code. The exercises also touch on other interesting areas of computer 
science such as games programming, data structures, algorithms and 
artificial intelligence.  

\subsection{Course structure}

\subsubsection{Lectures and seminars}

The course consists of two lectures per week (Monday 3-4pm in the L3 and 
Thursday 3-4pm in MS.01). The lectures are important. 
Even if you are already a competent programmer, the lectures will guide 
you on the things you will need to know for the exam and the coursework.
Lecture notes will be posted to the course website prior to the lectures allowing
you the opportunity to print hard copies to follow in the lectures. You may find that these do not 
make complete sense outside the scope of the lectures -- so don't miss the 
lectures!
 
{\bf The first lecture will take place on Thursday of week 1 in MS.01. }
 
You are also required to go to one seminar per week\footnote{Don't be 
confused by the fact that your timetable might have more than one CS118 
seminar slot marked on it. All this means is that they have printed all the 
seminar times, which session you will attend will be announced in due course.}. 
The seminar groups are posted in the reception area of the computer science 
building (and also on the course web page); you 
should check the lists to find out when your first seminar is. 
{\bf The seminars are scheduled to begin on Thursday of week 1 (from 5pm).}
 
The seminars serve two purposes. Firstly they allow you to interact with a
seminar tutor who will be an expert in the subject area and who will be able to 
guide you through any difficult moments. Make good use of your seminar tutor
as they are there to assist you and answer any questions which you may have 
concerning this course. Secondly the seminars are designed to test you, to 
make sure that you are keeping up with the course material. Throughout the term there 
will be four problem sheets released for you to complete. The seminars are a good opportunity
to get assistance on any aspect of these problem sheets or the coursework.
 
The seminars and problem sheets are {\bf not} optional. Each of you will be 
monitored to make sure that you are completing the problem sheets and as 
long as you do this you will receive some credit. Your overall mark for this module
will be based on your Exam, your two coursework assignments and your problem sheets.

\subsubsection{Coursework}

This course is not merely an introduction to programming. Rather it is 
intended to give you a modest taste of what problem solving using computer 
software is all about: combining creativity with a rigorous, analytic 
approach to produce an end result which can be relied upon to achieve its 
design goals. The fact that you will learn about the Java programming 
language is simply an added bonus. This being the case, it is not the 
purpose of the course to teach you about all the facilities of 
the Java programming language; you may decide that further reading is 
useful to your studies, but the assessment of your programs will be based on
their correctness and clarity, rather than their complexity.  
 
There are two pieces of coursework. The first coursework must be complete 
by \deadlineone; the second must be complete by 
\deadlinetwo. Your work will be assessed as follows:

{\it You will be required to come to room CS0.06 in the Computer Science 
Department at set times on the marking days to demonstrate 
your working program; here you will be interviewed, during which you will 
be asked to explain how your code works
as well as being tested on some of the fundamentals of the coursework.}

This may sound a little heavy, but I can assure you that if you have done 
the necessary study and produced your own independent solutions then you will not 
have any problems.

A record of your work will be stored by the University of Warwick 
Tabula online submission system. This software allows us to record your 
progress, justify the mark which you were awarded and detect plagiarism.

More information regarding Tabula can be found at 

{\tt http://tabula.warwick.ac.uk/}

\subsubsection{Adding it up}

The coursework and seminars will account for 40\% of you final 
mark for this course. The remaining 60\% of the marks are taken from your 
two-hour end of year exam which usually takes place in week 1 of term 3.

\subsection{Computing support}

\subsubsection{User codes}

Before you start programming, you need to get hold of a user code. 
Rather confusingly you will be allocated two user accounts, one by the 
University IT services and one by the Department of Computer 
Science.
 
The first thing you should do is get your IT services account.
The University of Warwick's Online Enrolment System is now well established.  
This should mean that when you collect your enrolment
certificate (by going to {\tt www.warwick.ac.uk/enrolment}) you will also be asked to complete the on-line Computer
Use Registration form, and as a result will receive a username and password. 
This will give you access to the University's IT network and facilities; it 
will not however give you access to the computers in the Computer Science
Department. 
 
Sometime after setting up your IT services user code and account, 
you will be emailed with your Computer Science account details\footnote{You 
should expect this to be done before your first CS118 seminar.}. You should 
therefore {\bf check the mail on your IT services account and when these details 
arrive you should make a note of your CS user code and password and bring these
with you when you come to your first seminar}. You are then ready to log on to the 
machines in Computer Science. Please remember that the IT services and Computer 
Science accounts are different and 
although they might have the same user code, they may well have 
different passwords. Please try not to forget your password, you will find 
that it is a real pain if you do.
 
If all this business of getting an account seems too confusing to be true, 
do not panic. Ask around to see if anyone else has worked it out and talk 
to them. Failing that, ask one of the tutors. Failing that, ask me.

\subsubsection{Course web page}

All the information in this guide (and more) can be found on the course 
web page: 

{\tt http://go.warwick.ac.uk/cs118/} 

It is certainly worth bookmarking this page in your favourite web browser 
as it will be very useful as the course progresses. The course web page 
contains:

\begin{itemize}

\item
This {\em Guide} to CS118;

\item
All the lecture notes as they become available;

\item
Copies of the seminar sheets;

\item
A lecture summary, course overview and course syllabus;

\item
Trouble shooting information and handy hints for coursework and 
seminars;

\item
General systems information (including guides to UNIX etc.);

\item
Java resources including the robot-maze software.

\end{itemize}

Do take a look at the course web page from time to time as it will be updated 
periodically. 

\subsubsection{Java sources}

Both the IT services computers and also the Computer Science computers have 
copies of Java running on them. If you want to get hold of a copy for your
home computer then this is very easy. A download can be found at: 

{\tt http://www.oracle.com/technetwork/java/javase/downloads/} 

There are Microsoft, Linux and macOS versions of the software at this site
and installing the software should be relatively straight forward and 
should not take you more than about ten minutes.
Make sure you download the {\em Java Development Kit (JDK)}.
 
To run Java on a Windows laptop, open a {\em Command prompt} window that you 
should be able to find on most versions of Windows. The advantage of this is 
that the set-up then looks just like a terminal window on one of the computers in the 
Department and all the commands that you type in to each are the same. 
You will probably find that Java installs to the directory 
{\tt C:$\backslash$Program Files$\backslash$Java$\backslash$jdk1.8.X$\backslash$bin}.  
 
The Java package for the robot-maze software which you will use during your 
coursework can be found on the course web page. Instructions for building your code
with the robot-maze software will be set out in due course.

\subsection{Books}

Programming can be complicated business, but the internet
should have plenty of information to ease the learning process.
However, some of you may prefer to learn from a introductory text book.
Each year I try to select the most appropriate Java books for this course. 
In making the selection I try to ensure that the books will be understandable,
will reflect the range of talents and abilities of the people on the course,
and will be useful later in your degree. 

If I were going to buy a good book from new then I would pick up one
of:
\begin{itemize}

\item
{\em Introduction to Java Programming Comprehensive} 
by Y Daniel Liang, published by Prentice Hall, ISBN 0-13-222158-6 
(\pounds 48.88 on Amazon.co.uk or second-hand for \pounds 23)

\item 
{\em Beginning Java Programming -- The Object-Oriented Approach} by 
Bart Baesens, Aim\'{e}e Backiel and Seppe vanden Broucke, published by 
Wrox, ISBN 978-1-118-73949-5 (\pounds 19.51 on Amazon.co.uk)

\item
{\em Java Programming} by Joyce Farrell, published by Cengage Learning, 
ISBN 978-1-285-85691-9 (\pounds 80.99 on Amazon.co.uk)

\end{itemize}

The Liang book follows the lectures very closely so is your best bet if 
you are feeling a bit unsure about this programming lark. 
 
The other two text books are both recently updated introductory texts and as such
should contain the most up-to-date information. They are also text books
I have recently been sent free of charge so I'm recommending them as a
courtesy and because I can easily look up any exercises should you be stuck
on one and require help! 
 
You may already have a Java book. I guess the only reason that you might not 
want to use this is if it is using Java 1.0. This is now pretty old and there
are enough differences to warrant splashing out and buying a new book.
 
To be honest most of the information that you are going to need can be 
found on the Web. The only difficulty with this is that it is hard to see 
the wood for the trees. So a book on programming might
be the easiest way to avoid all the noise that you get on the Web.  

\subsection{Practicalities}

If you have any problems regarding CS118 matters you can come and find me in 
my office (CS2.02) in the Computer Science department.\\
 
My email address is 
{\tt steven.wright@warwick.ac.uk}. 
 
You will probably find that many of the issues which concern you are also
experienced by others on the course. Try talking to your friends and seminar
tutors, it is likely they will either have similar problems or will have solutions to
these issues.

\subsection{And if it all starts to go wrong?}

You may get to a point in this course where you just don't know what is going 
on any more. The golden rule is not to panic or get depressed. There are a number of procedures in place to
catch you before you fall, but you have to be pro-active in finding help.
 
The seminar tutors are usually pretty good at answering questions. You can 
email them or find them out of hours by asking the receptionist in Computer Science where 
their offices are. As long as you don't pester them every five minutes, I am
sure they would be pleased to help out. If you can not find them, you can of
course come and find me. 

\subsection{Acknowledgements}

The robot-maze software which you will use in your coursework has an 
interesting history. It started life at 
Queen Mary College, University of London as the brainchild of Ian Page. 
When Ian moved to Oxford he and Colin Turnbull made extensive rewrites and
the robot-maze to this day exists as a means by which engineering students
learn the C programming language. Kevin Parrott, Alison Noble, 
Andrew Zisserman and Prof. Stephen Jarvis ran this software largely untouched 
at Oxford for a number of years. The new Java version of the 
software is thanks to Phil Mueller from the University of Warwick (now 
himself at Oxford), and the new exercises are courtesy of Ioannis Verdelis
(formerly of Warwick and now at Manchester University). I think 
you will agree that the result is a 
wonderfully interactive way in which to learn the art of programming.
