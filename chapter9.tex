\clearpage
\section{Coursework 2 (Part 2): \\
         The Grand Finale}

The exercises in this chapter are designed to test the very best of you. 
Although it should be said that it is 
possible to get a good grade in the practical component of the CS118 
course without having done the exercises in this chapter; 
this means that if you do 
not get a chance to finish this part then you should not panic. You may 
however find the exercises interesting and decide that you have time to 
attempt some or all of the questions provided. Even if you do not complete 
the exercises, you might get some marks for trying. So even if your robot 
is a bit wonky, you may still gain credit for a part solution. \\

\noindent
In this second part of Coursework 2 you are required to design, build and 
test a {\em learning robot}. You will receive much less step-by-step 
guidance and as a result I expect to see lots of exciting and innovative 
solutions. Your solutions will be marked on design, programming style and 
of course correctness\footnote{If your program works then you can be 
assured of at least 60\% of the marks, if you have a smart design then 
you will bank a further 20\%, and if you have programmed like a coding god 
then you can expect the remaining 20\% of the marks.}.\\

\noindent
The programs which you will write for this section are probably the most 
interesting in this course. You will produce some very clever robot 
controllers as answers to the questions. It can also be very satisfying to 
watch the more advanced robots in action. \\

\noindent
As a consequence of the work being more difficult, you will receive extra 
credit for any work you do from this chapter. It is very difficult to quantify
exactly what this means in terms of {\it your} marks, as other factors such as 
programming style, reusability, etc. will play a part. However, you might 
like to think of these exercises as constituting the difference between an 
{\em A} grade and a {\em B} grade in your coursework (depending on 
where the grade boundaries are drawn). \\

\noindent
This year there will also be a prize for the person who develops the 
best solution. You could win this by going beyond the specification, or
perhaps you could produce the shortest solution to the grand finale\footnote{The current
record for shortest solution is 5 lines of code, including all of the Java filler! Though a submission
of 30 lines or below would be very impressive for first year students.}. Previous
winners have also turned the maze environment into a game, and created interesting 
graphical displays to show the maze-tree as it was being explored.

\subsection{The {\it Grand Finale}}

You receive a call from NASA. They have seen your robot in action and plan 
to use some of your code in their next mission\footnote{Heaven forbid.}. 
However, while it is clear that your robot searches in a very sophisticated 
way, NASA point out that it does not learn from its mistakes\footnote{Unlike
NASA, of course.}. What they would like is a robot that learns. \\

\noindent
$\diamondsuit$ The aim of this last part is to build a {\bf learning robot} that can use 
information gathered during previous runs through a maze to find a target at 
increasing speed. \\

\noindent
The plan is to build on your previous solution. The robot will
search the maze (in its {\bf Explorer}-like way) the {\em first time} 
the robot is run through a maze but, 
the {\em second time}\footnote{or more} the robot is run, it will use its 
virtual map (its memory if you like) to find the target more quickly.
What you would expect the robot to do the second time round is exclude the 
routes through the maze which went nowhere and instead select those which
it knows will take it towards the target. \\

\begin{figure}[t]
\centering
\includegraphics[width=4.5in]{finale1.eps}
\caption{A trace of the {\bf GrandFinale} robot the first run through the 
maze.
\label{finale1}}
\end{figure}    

\begin{figure}[t]
\centering
\includegraphics[width=4.5in]{finale2.eps}
\caption{A trace of the {\bf GrandFinale} robot the second time it 
runs through the maze.
\label{finale2}}
\end{figure}    

\noindent
An example of this behaviour is demonstrated in Figures~\ref{finale1} 
and~\ref{finale2}. In Figure~\ref{finale1} we see the trace of the robot the 
first time it is run through a fairly extensive maze. As you would expect it 
is fairly thorough about the areas which it explores, 
finally however it reaches the target.\\

\noindent
When the robot is run again, shown in Figure~\ref{finale2}, the robot can 
use the information which it stored about the maze during the first run to 
direct its search for the target. As you see it needs far less exploration 
the second time round. The aim of the second part of Coursework 2 is to 
model this behaviour. \\

\noindent
There is more than one way of doing this and in order to provide you with 
some (modest) assistance I shall describe two approaches which you might 
like to take. I do not mind which of these you go for, if any.

\subsection{Route A}

$\diamondsuit$
Modify {\bf Explorer} so that it records the junctions (if any) 
that each known junction's exit(s) lead to. Test that your program stores the
correct information by temporarily modifying it to print the information on
the screen and comparing the results with the layout of small test mazes.
The information gathered should be retained between runs in the same maze.\\

\begin{figure}
\centering
\includegraphics[width=3.8in]{junc_info.eps}
\caption{Storing more junction information.\label{7.1}}
\end{figure}

\noindent
You may wish to experiment with the use of complex data structures
or arrays of arrays
to record this information in a conveniently accessible form. Figure~\ref{7.1}
shows the sort of data structure which you will need and the extra information
which you are likely to store. Note that as well as storing the 
direction from which the robot entered a junction, the controller also stores 
the junction 
which taking a {\tt LEFT}, {\tt RIGHT}, {\tt BEHIND} or {\tt FORWARD} path 
would lead to. \\

\noindent
In the figure you will see that the array index at which the information relating
to a particular junction is stored, provides a very
convenient means of identifying that junction. Since these identifying
integers will be positive (or zero), exits which have yet to be explored can
be represented as a negative number. Remember that a method is
provided which tells you whether you are computing the first run
in a maze, and using this you can detect whether the maze has changed since 
the previous runs. \\

\noindent
The second task is to
design, build and test a method which uses the information collected by
the controller to compute a route between the start and end of the maze. \\

\noindent
My advice here is to think first and implement second! Consider how you might
represent and compute such a route. Give yourself time to think. Go away from
the computer, with pencil and paper (or coffee, or beer, whatever...) to 
design a good scheme. Then try to implement and test your ideas. If inspiration
fails to strike even after thinking hard, your seminar tutor will be able to
offer a hint or two. There are quite a number of possible methods which you could
use, and each can be implemented as a computer program in many different
ways. So, don't panic if your solution sounds completely different to that 
offered by the seminar tutor -- the chances are that you are both right. \\

\noindent
Save your working implementation to this exercise in the file 
{\tt GrandFinale.java}.

\subsection{Route B}

$\diamondsuit$
There is a solution to this problem which many perceive as simpler to that 
presented in Route A. This route is based on your answer to Exercise 2 and 
so you might like to remind yourself what that was about.\\

\noindent
It is possible to store the arrived-from direction and not the x- and 
y-coordinates of the robot and still build a learning robot. What you need to 
do in this case is treat the arrived-from directions as a {\em stack} of 
values. \\

\noindent
The analogy which is often introduced when describing stacks in Computer 
Science is a pile of dinner plates. If you imagine this pile, then plates can
only be introduced at the top of the pile (if you want to avoid any lifting)
and similarly taking a plate (in this lazy way) means taking one from 
the top as opposed to anywhere else in the pile. This method of putting 
things on and taking things off is described as {\em last in first out}, and 
Computer Scientists use the terms {\em push}ing items onto the stack and 
{\em pop}ing them off. \\

\noindent
You can use a stack to record the arrived-from directions of the robot. Each
time the robot arrives at a junction the arrived-from direction should 
be pushed onto the stack; when the robot is backtracking the arrived-from 
directions should be popped off the top of the stack. \\

\noindent
If you use this approach you will find that by the time the robot reaches
the target the stack will contain a route to the target. The trick is to 
then use this stack the next time round to direct the robot straight to the 
target.\\

\noindent
Save your working implementation to this exercise in the file 
{\tt GrandFinale.java}.

\subsection{Submitting your coursework}

The files which should be submitted for the second coursework are {\tt Ex1.java},
{\tt Ex2.java}, {\tt Ex3.java} and also {\tt GrandFinale.java}.\\

\noindent
In order to make the marking of your work easier, you should make sure that the
class names correspond to the file names -- that is, file {\tt Ex2.java} 
contains the definition \\

{\tt public class Ex2} \\

\noindent
and similarly for the other files. \\

\noindent
Submit your working using Tabula, as before. The marking of this coursework will take place on
\deadlinetwo. Be there or be square\footnote{In fact, be there 
or get no marks.}.

\subsection{Epilogue}

This coursework has touched on a number of different areas of Computer 
Science. You have learnt something about {\em specifications} and {\em 
refinement}, you have learnt something about {\em programming} and 
{\em software testing}. You will have also touched on {\em data structures}
and {\em algorithms}, and also {\em AI}. \\

\noindent
There is much to learn in the remainder of your degree course but this 
should give you an idea as to how all these areas fit together and how 
important each is in its own right. \\

\noindent
Programming is a complicated business and you are not going to have mastered 
it in ten weeks. It is a bit like learning to drive -- you have probably 
crashed a few times already, or at least come off the road -- and you 
will get better the more you do. By the summer many of you will be taking 
up well paid summer jobs fixing peoples' Java code. This may sound hard to 
believe right now, but each year it is the same; the only thing that 
changes is that the wages go up!
