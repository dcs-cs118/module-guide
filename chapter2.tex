\pagebreak
\section{Getting started}

This chapter is intended to be used as part of the first lab session to get you started with some of the tools you will be using throughout this module and your time at Warwick. It is important that you know where you should be each week. For your first lab you should meet in the Atrium of the Computer Science department building. You lab tutor or one of the module organisers will meet you and take you into one of the labs where you will spend the hour working on some tasks. If you are having problems with user accounts then this is a chance to sort these out. In subsequent labs you are welcome to enter the lab, log in and get started.

\subsection{Linux}

All the computers in the Computer Science building run the Red Hat ``flavour'' of Linux. Other ``flavours'' are available for
your laptop (e.g. Ubuntu). You will certainly need to pick up some Linux skills while you are at Warwick. One way to learn more is by attending the CS133 Professional Skills lectures.

\subsection{Tools}

All the machines in the department's labs have everything installed you need to write and run Java programs. In this section, we summarise some of the key tools you will be using and explain how you might also be able to use them on your own laptop or PC at home.

\textbf{TODO}: write about JRE, text editors, gradle, etc.

\subsection{Editing, compiling, and running Java code}

\task{In a text editor of your choice, create a new file which contains the following text:}

\begin{minted}{java}
public class Hello { 
    public static void main(String[] args) {
        System.out.println("Hello World!");
    }
}
\end{minted}

Make sure you copy the program exactly (the capitalisation is important). Save the program as \texttt{Hello.java} (again, capitalisation is important!) in a folder of your choice.

\taskLine

\task{Now compile the program. To do this you need to open a terminal window, navigate to the folder where you saved \texttt{Hello.java} and then run (we use \mintinline{bash}{$} throughout this guide to denote a terminal prompt):} 

\begin{minted}{bash}
$ javac Hello.java
\end{minted}

If there is no output to the terminal, then the program has been compiled successfully and a file named \texttt{Hello.class} has been generated in the current directory. If the Java compiler reports any errors at this point, however, then you may have typed the program incorrectly. Correct any errors in the program by switching back to the text editor, making corrections, and then running the command above again until there is no output and \texttt{Hello.class} is generated.

\taskLine

\task{You can now run the program by typing:} 

\begin{minted}{bash}
$ java Hello
\end{minted}

The program should display {\tt Hello World!} in the terminal. Congratulations! You have just learnt how to edit, compile, and run a Java program.

\taskLine

\task{Edit the {\tt Hello.java} program again and change the message
so that it says something else. You should be able to do this
without knowing any Java. Save, compile and run it. }

If at any point you get stuck then ask for some help.

\taskLine 

\task{Create a new file called \texttt{Age.java}. Type the following
program exactly as it is written:}

\begin{minted}{java}
import java.util.Scanner;

public class Age { 
    public static void main(String[] args) {
        Scanner sinput = new Scanner(System.in);
        System.out.println("Enter your age:");
        int age = sinput.nextInt();
        int doubleAge = age * 2;
        System.out.println(
          "How old? That's half way to " + 
          doubleAge + "!");
    }
}
\end{minted}
You should note that when you make a new Java file it must
always end with the \texttt{.java} file extension and the filename must match the name which usually follows the \mintinline{java}{public 
class} part in the contents of the file (including matching capitalisation!). 

You may also notice that the formatting of the lines in the example does not really matter and statements can be spread out over several lines. Java uses semicolons to understand where statements end and curly braces to denote where a block of code starts and ends.

\taskLine 

\task{The meaning of the individual components of the program above will become clearer as you learn more about programming in Java. However, try to experiment with the different parts of the program to see if you can work out what each of them does.}

\taskLine 

\subsection{What next?}

If you have gone through the exercises above and have made sure to sign in with one of the teaching assistants then you are free to go.

A useful thing to do if you get to the end of these exercises way before the end of the lab is to look at the next chapters of this guide. If you decide not to do this now, then you should put this on your to-do list to be completed by the end of week 2 (latest).
