\pagebreak
\section{Getting Started}

This chapter is intended for use during your first seminar session. Seminars 
will take place in the programming labs. It 
is important that you know where you should be each week. {\bf For your first 
seminar you should meet in the reception area of the Computer Science 
department building}. You seminar tutor will meet you and take you into 
one of the labs where you will spend the hour working through 
the exercises in this chapter. If you are having problems with user accounts 
then this is a chance to sort these out. 

In subsequent labs you are welcome to enter the lab, log in and get started.

\subsection{UNIX}

All the computers in the Computer Science building run the Linux operating 
system. We use the Red Hat `flavour' of Linux, though others are available for
your laptop (e.g. Ubuntu). Linux is a UNIX-like operating system commonly used
in computing industries, especially on server systems where stability and security
are key. 

You will certainly need to pick up some UNIX skills while you are at 
Warwick. One way to start is to buy a copy of the excellent 
{\em Introducing UNIX and Linux} by the very distinguished authors Joy, 
Jarvis and Luck. \\

$\diamondsuit$ 
I would suggest that you have a look at this book and study the chapter titled
`Getting Started', though I would personally ignore the 
direction to the text editor known as {\em vi} and also the section on 
electronic mail using {\em mailx}.

\subsubsection{Login}

When you log in to the computers you will enter a windows-like
environment. There are a number of other {\em Session} environments that you
can select from the log-in screen, but this guide is written from the 
point of view of the {\em Default} session and so I can assume no 
responsibility for problems outside of this domain. 

\subsubsection{Mail}

$\diamondsuit$
You do not have a mailbox on the CS system. All email will be sent to your
University mail account, which you can access from the web using the University's
live mail service: 

{\tt http://go.warwick.ac.uk/mymail}

\subsubsection{The Internet}

$\diamondsuit$
You can invoke a web browser
(the default is Google Chrome) by navigating through the menu in the bottom left-hand corner.

While you are here you should find the course web-page. 

{\tt http://go.warwick.ac.uk/cs118/} 

I would suggest that you {\em Bookmark This Page}.

{\bf Any important announcements will be made in lectures and posted to the course webpage.}

\subsubsection{Editing files}

Red Hat contains a number of text editing tools. One of these, called
{\em kwrite}, is available from the desktop and will be our editor of 
choice. Other editing tools can be found from the Red Hat window manager, or
run from a terminal window. When working remotely in a terminal environment, you
may like to try out {\em nano}.

\subsection{Editing, compiling and running Java code}

$\diamondsuit$ 
Create a new file which contains the following text:

\begin{verbatim}
public class Hello { 
    public static void main(String[] args) {
        System.out.println("Hello World!");
    }
}
\end{verbatim}

Make sure you copy the program exactly (the capitalisation is important). 
Save the program as {\tt Hello.java} (again, capitalisation is important!) 
in a folder of your choice (Note: you
will need to navigate to this folder in the terminal).

$\diamondsuit$ 
Now compile the program. To do this you need to open a terminal window 
(select the screen button from the window manager) in which you should type: 

{\tt javac Hello.java} 

If the Java compiler detects any errors, then you must have
typed the program incorrectly. Correct any errors in the program
by re-entering the editor, making corrections, and then
recompiling. 

$\diamondsuit$ 
Run the program by typing: 

{\tt java Hello} 

The program should display {\tt Hello World!} Congratulations!
You have just learnt how to edit, compile and run a Java
program.

\subsubsection{Some easy programs}

$\diamondsuit$ 
Edit the {\tt Hello.java} program again and change the message
so that it says something else. You should be able to do this
without knowing any Java. Save, compile and run it. 

{\bf If at any point you get stuck then ask for some help.}

$\diamondsuit$ 
Create a new file called {\tt Age.java}. Type the following
program exactly as it is written:

\begin{verbatim}
  import java.util.Scanner;
  public class Age { 
    public static void main(String[] args) {
       Scanner sinput = new Scanner(System.in);
       System.out.println("Enter your age:");
       int age = sinput.nextInt();
       int doubleAge = age * 2;
       System.out.println("How old? That's half way to " + doubleAge + "!");
    }
  }
\end{verbatim}
 
You should note that when you make a new Java file it must
always end with {\tt .java} and the name which follows the {\tt public 
class} in the program file must be the same name as the file (including 
matching capitalisation!). 

This program uses the {\tt nextInt} method of the
{\tt java.util.Scanner} class. This class makes getting input from the 
terminal much simpler. It first creates a {\tt Scanner} object, using the 
Systems input stream ({\tt System.in}), then uses a special method to read
an integer into a variable.

$\diamondsuit$ 
The marks for this course might finally be weighted as follows: 
Coursework 30\%; problem sheets 10\%; final exam 60\%. 

Write a program called {\tt TotalMark.java} that reads in the
three marks (each out of 100) and writes out the total weighted
mark\footnote{There are a couple of things which you might like to 
investigate for this question -- try and find out more about {\em Numeric
Datatypes} and {\em Numeric Operators} from the lecture notes or textbooks 
etc.}. You should not have to create more {\tt Scanner} objects as you can use
the same one multiple times (with successive calls to {\tt nextInt}).


\subsection{Some UNIX commands}

In case you have not had time to learn some UNIX skills yet, here are some useful 
commands that you might like to type into the terminal window. Write next 
to each what you think the command does: 

\indent
{\tt ls} 

\indent
{\tt mkdir test} 

\indent
{\tt cd test} 

\indent
{\tt cd ..} 

\indent
{\tt rmdir test} 

\indent
{\tt ls -al} 

\indent
{\tt man ls} 

\indent
{\tt passwd} 

Use this last command VERY carefully. You should certainly aim to set your
password to something memorable in this session -- if you are not yet 
ready to do this, then type Ctrl-C into the 
terminal window, and this will kill the command. 


\subsection{Web-based course material}

$\diamondsuit$ 
All the course material is available on the Web. Get a web browser
running on your computer and take a look at the page: 

{\tt http://go.warwick.ac.uk/cs118/} 

If you've missed the first 2 instructions to add this page to your bookmarks, do it now! 

It is strongly recommended that you all spend several hours at a
terminal during the first two weeks of term getting familiar with
Linux, the text editor etc. In any event, you should 
also be spending
several hours per week writing Java programs. Copy examples out
of your textbooks, customise them for your own use and experiment
by trying to write programs of your own. {\bf Learning to program is
impossible without the practical experience that is gained only by
sitting down at a computer and doing it.}

\subsection{What next?}

$\diamondsuit$
This is the end of the first seminar session and if you have gone through the 
work and signed in with your seminar tutor then you are free to go.

A useful thing to do if you get to the end of these exercises way before 
the end of the seminar hour is to look at Chapter 5, and also the first part of Chapter 6. 
This introduces you to the programming environment which you will use for 
your coursework. 
 
If you decide not to look at this now, then you should put this on your to-do
list to be completed by the end of week 2 (latest).
